\documentclass[10pt]{article}
% The template was originally created by Stephanie Hicks.
% Modifications by Will Townes

\usepackage{tablefootnote}
\usepackage{calc}
\usepackage{paralist} % This gives us fun enumeration environments. compactitem will be nice.
\usepackage{enumitem}

\reversemarginpar % Layout: Puts the section titles on left side of page

%% Use these lines for letter-sized paper
\usepackage[paper=letterpaper,
            %includefoot, % Uncomment to put page number above margin
            marginparwidth=1.0in,     % Length of section titles
            marginparsep=.05in,       % Space between titles and text
            margin=.70in,               % 1 inch margins
            includemp]{geometry}

\setlength{\parindent}{0in} %% More layout: Get rid of indenting throughout entire document

\usepackage{fancyhdr,lastpage}
\pagestyle{fancy}
%\pagestyle{empty}      % Uncomment this to get rid of page numbers
\fancyhf{}\renewcommand{\headrulewidth}{0pt}
\fancyfootoffset{\marginparsep+\marginparwidth}
\newlength{\footpageshift}
\setlength{\footpageshift}
          {0.5\textwidth+0.5\marginparsep+0.5\marginparwidth-2in}
\lfoot{\hspace{\footpageshift}%
       \parbox{4in}{\, \hfill %
%                    \arabic{page}  of \protect\pageref*{LastPage} % +LP
                    \arabic{page}                               % -LP
                    \hfill \,}}

% Finally, give us PDF bookmarks
\usepackage{color,hyperref}
\definecolor{darkblue}{rgb}{0.0,0.0,0.3}
\hypersetup{colorlinks,breaklinks,
            linkcolor=darkblue,urlcolor=darkblue,
            anchorcolor=darkblue,citecolor=darkblue}


%%%%%%%%%%%%%%%%%%%%%%%%%%% Helper Commands %%%%%%%%%%%%%%%%%%%%%%%%%%%%

% The title (name) with a horizontal rule under it. e.g. Usage: \makeheading{name}
\newcommand{\makeheading}[1]%
        {\hspace*{-\marginparsep minus \marginparwidth}%
         \begin{minipage}[t]{\textwidth+\marginparwidth+\marginparsep}%
                {\large \bfseries #1}\\[-0.15\baselineskip]%
                 \rule{\columnwidth}{1pt}%
         \end{minipage}}

% The section headings. e.g. Usage: \section{section name}
\renewcommand{\section}[2]%
        {\pagebreak[2]\vspace{1.3\baselineskip}%
         \phantomsection\addcontentsline{toc}{section}{#1}%
         \hspace{0in}%
         \marginpar{
         \raggedright \scshape #1}#2}

% An itemize-style list with lots of space between items
\newenvironment{outerlist}[1][\enskip\textbullet]%
        {\begin{itemize}[#1]}{\end{itemize}%
         \vspace{-.6\baselineskip}}

% An environment IDENTICAL to outerlist that has better pre-list spacing
% when used as the first thing in a \section
\newenvironment{lonelist}[1][\enskip\textbullet]%
        {\vspace{-\baselineskip}\begin{list}{#1}{%
        \setlength{\partopsep}{0pt}%
        \setlength{\topsep}{0pt}}}
        {\end{list}\vspace{-.6\baselineskip}}

% An itemize-style list with little space between items
\newenvironment{innerlist}[1][\enskip\textbullet]%
        {\begin{compactitem}[#1]}{\end{compactitem}}

% An environment IDENTICAL to innerlist that has better pre-list spacing
% when used as the first thing in a \section
\newenvironment{loneinnerlist}[1][\enskip\textbullet]%
        {\vspace{-\baselineskip}\begin{compactitem}[#1]}
        {\end{compactitem}\vspace{-.6\baselineskip}}

% To add some paragraph space between lines.
% This also tells LaTeX to preferably break a page on one of these gaps
% if there is a needed pagebreak nearby.
\newcommand{\blankline}{\quad\pagebreak[2]}

% Uses hyperref to link DOI
\newcommand\doilink[1]{\href{https://dx.doi.org/#1}{#1}}
\newcommand\doi[1]{doi:\doilink{#1}}
\newcommand\arxiv[1]{\url{https://arxiv.org/abs/#1}}


%%%%%%%%%%%%%%%%%%%%%%%% End Helper Commands %%%%%%%%%%%%%%%%%%%%%%%%%%%


%%%%%%%%%%%%%%%%%%%%%%%%% Begin CV Document %%%%%%%%%%%%%%%%%%%%%%%%%%%%

\begin{document}
\makeheading{F. William (Will) Townes \hfill \small{Updated \today}}

\section{Contact Information}
%
% NOTE: Mind where the & separators and \\ breaks are in the following
%       table.
%
% ALSO: \rcollength is the width of the right column of the table
%       (adjust it to your liking; default is 1.85in).
%
\newlength{\rcollength}\setlength{\rcollength}{2.05in}%
%
\begin{tabular}[t]{@{}p{\textwidth-\rcollength}p{\rcollength}}
Department of Statistics and Data Science & website: \href{http://willtownes.github.io}{willtownes.github.io} \\
Carnegie Mellon University & email: \href{mailto:ftownes@andrew.cmu.edu}{ftownes@andrew.cmu.edu} \\
Pittsburgh, PA USA & \\  %twitter: @will_townes \\
\end{tabular}

%%%%%%%%%%%%%%%%%%%%%%%%%%%%%%%%%%%%%%%%%%
%%%%%%%%%% Interests %%%%%%%%%%%%%%%%%%
%%%%%%%%%%%%%%%%%%%%%%%%%%%%%%%%%%%%%%%%%%

\section{Summary}
I am an applied statistician working primarily with biomedical data. I am involved with infectious disease as part of the Delphi Group and have developed methods for spatial and single-cell genomics. I am also interested in wearable devices, ecology, and economics. I strive to produce open source software implementations to help scientists achieve their research goals more effectively.

%%%%%%%%%%%%%%%%%%%%%%%%%%%%%%%%%%%%%%%
%%%%%%%%%%%%%% Professional Experience %%%%%%%%%%%%
%%%%%%%%%%%%%%%%%%%%%%%%%%%%%%%%%%%%%%%
\section{Professional Experience}
\vskip -.25in
\begin{table}[!htb]
\begin{tabular}{ll}
2022 - present & \textbf{Assistant Professor}, Department of Statistics and Data Science\\
&Carnegie Mellon University, Pittsburgh, PA\\
2019 - 2022 & \textbf{Postdoctoral Researcher}, Department of Computer Science\\
&Princeton University, Princeton, NJ (advisor: Barbara Engelhardt)\\
2011 - 2014 & \textbf{Software Test Engineer}, SRA International, Arlington, VA   \\
2008 - 2010 & \textbf{Software Analyst}, Perrin Quarles Associates, Charlottesville, VA   \\
2007 - 2008 & \textbf{Fulbright Scholar}, University of the Philippines \\
& Conducted independent forest ecology research while living with an \\
& indigenous \textit{Aeta} community in Bataan Province. \\
\end{tabular}
\end{table}

%%%%%%%%%%%%%%%%%%%%%%%%%%%%%%%%%%%%%%%
%%%%%%%%%%%%%% Education %%%%%%%%%%%%%%%%%%
%%%%%%%%%%%%%%%%%%%%%%%%%%%%%%%%%%%%%%%
\vskip -.2in
\section{Education}
\vskip -.25in
\begin{table}[!htb]
\begin{tabular}{cl}
2019 & \textbf{Ph.D. Biostatistics}, Harvard University, Cambridge, MA USA \\
& \textit{Dissertation Committee: Rafael Irizarry (Dana-Farber Cancer Institute, Data Science)} \\
& \hspace{1.5in} \textit{Jeff Miller (Harvard University, Biostatistics)} \\
& \hspace{1.5in} \textit{Martin Aryee (Massachusetts General Hospital, Pathology)} \\
%& \textit{Dissertation: Probabilistic Models for Genetic and Genomic Data with Missing Information} \\
% & \\
% 2016 & \textbf{A.M. Biostatistics}, Harvard University, Cambridge, MA USA \\
2013 & \textbf{M.S. Mathematics \& Statistics}, Georgetown University, Washington, DC USA \\
2007 & \textbf{B.S. Biology}, Washington \& Lee University, Lexington, VA USA
\end{tabular}
\end{table}

%%%%%%%%%%%%%%%%%%%%%%%%%%%%%%%%%%%%
%%%%%%%%%%% Publications %%%%%%%%%%%%%%%%%%
%%%%%%%%%%%%%%%%%%%%%%%%%%%%%%%%%%%%
\vskip -.2in
\section{Research}
\textbf{In Review}
\begin{enumerate}[label= {[\arabic*]}]
\item Jones A, {\bf Townes FW}, Li D, Engelhardt BE. Alignment of spatial genomics and histology data using deep Gaussian processes (2022). Preprint: \doi{10.1101/2022.01.10.475692}.
\item Verpeut JL, Bergeler S, Kislin M, {\bf Townes FW}, Klibaite U, Dhanerawala ZM, Hoag A, Jung C, Lee J, Pisano TJ, Seagraves KM, Shaevitz JW, Wang SSH. Cerebellar contributions to a brainwide network for flexible behavior (2021). Preprint: \doi{10.1101/2021.12.07.471685}.
\end{enumerate}

% \textbf{Preprints}
% \begin{enumerate}[label= {[\arabic*]}]
% \end{enumerate}
\textbf{In Press}
\begin{enumerate}[label= {[\arabic*]}]
\item {\bf Townes FW}, Engelhardt BE. Nonnegative spatial factorization. {\it Nature Methods} (2022). \newline Preprint: \arxiv{2110.06122}.
\end{enumerate}

\textbf{Peer-reviewed Journal Articles}
\begin{enumerate}[label= {[\arabic*]}]
\item Gewirtz ADH, {\bf Townes FW}, and Engelhardt BE. Telescoping bimodal latent Dirichlet allocation to identify expression QTLs across tissues. {\it Life Science Alliance} (2022).\newline \doi{10.26508/lsa.202101297}.
\item Jones A, {\bf Townes FW}, Li D, Engelhardt BE. Contrastive latent variable modeling with application to case-control sequencing experiments. {\it Annals of Applied Statistics} (2022). \newline \doi{10.1214/21-AOAS1534}.
\item Hecker J, {\bf Townes FW}, Kachroo P, Lasky-Su J, Ziniti J, Cho MH, Weiss ST, Laird NM, Lange C. A unifying framework for rare variant association testing in family-based designs, including higher criticism approaches, SKATs, and burden tests. {\it Bioinformatics} (2020). \newline \doi{10.1093/bioinformatics/btaa1055}.
\item {\bf Townes FW}, Carr K, Miller JW. Identifying longevity associated genes by integrating gene expression and curated annotations. {\it PLOS Computational Biology} (2020). \newline \doi{10.1371/journal.pcbi.1008429}.
\item {\bf Townes FW}, Irizarry RA. Quantile normalization of single-cell RNA-seq read counts without unique molecular identifiers. {\it Genome Biology} (2020). \doi{10.1186/s13059-020-02078-0}.
\item {\bf Townes FW}, Hicks SC, Aryee MJ, Irizarry RA. Feature selection and dimension reduction for single-cell RNA-Seq based on a multinomial model. {\it Genome Biology} (2019). \doi{10.1186/s13059-019-1861-6}.
\item Marsh DM, {\bf Townes FW}, Cotter K, Farroni K, McCreary K, Petry R, Tilghman J. Thermal Preference and Species Range in Mountaintop Salamanders and Their Widespread Competitors. {\it Journal of Herpetology} (2019). \doi{10.1670/18-110}.
\item Hicks SC, {\bf Townes FW}, Teng M, Irizarry RA. Missing Data and Technical Variability in Single-Cell RNA-Sequencing Experiments. {\it Biostatistics} (2018). \doi{10.1101/025528}.
\item Hecker, J, Xu X, {\bf Townes FW}, Fier HL, Corcoran C, Laird N, Lange C. Family-Based Tests for Associating Haplotypes With General Phenotype Data: Application to Asthma Genetics. {\it Genetic Epidemiology} (2017). \doi{10.1002/gepi.22094}.
\item Valeri L, Patterson-Lomba O, Gurmu Y, Ablorh A, Bobb J, {\bf Townes FW}, Harling G. Predicting Subnational Ebola Virus Disease Epidemic Dynamics from Sociodemographic Indicators. {\it PLOS One} (2016). \doi{10.1371/journal.pone.0163544}.
\item {\bf Townes W}. Seed dispersal of the genus Leea in forest patches of Bataan, Philippines. {\it Ecotropica } (2010).
\end{enumerate}
%\vskip .2in

\textbf{Technical Reports}
\begin{enumerate}[label= {[\arabic*]}]
\item {\bf Townes FW}. Review of Probability Distributions for Modeling Count Data. {\it arXiv} (2020).\newline \arxiv{2001.04343}.
\item {\bf Townes FW}. Generalized Principal Component Analysis. {\it arXiv} (2019).\newline \arxiv{1907.02647}.
\end{enumerate}

% \textbf{Class Projects and Other Unpublished Work}
% \begin{enumerate}[label= {[\arabic*]}]
% \item Townes FW, Comment L. Bayesian Methods for Dependent Data. Harvard Biostatistics 245 (Multivariate and Longitudinal Analysis). April 2016.
% \item Townes FW. Variational Inference for Mixtures of Dirichlet Network Distributions. Harvard Computer Science 282R (Bayesian Nonparametrics Seminar).
% \item Townes FW, Liu JZ. Bayesian Nonparametric Time Series Modeling. MIT 6.882 (Bayesian Modeling and Inference).
% \item Townes FW, Marquez-Luna C, Onnela JP. Network Connectivity and Infectious Disease Modeling.
% \item Townes FW, Hicks SC, Aryee MJ, Irizarry RA. Varying-Censoring Aware Matrix Factorization for Single Cell RNA-Sequencing. {\it bioRxiv} (2017). \doi{10.1101/166736}.
% \end{enumerate}
%\vskip .2in

\section{Talks}
\textbf{Invited}
\begin{enumerate}[label= {[\arabic*]}]
\item Biomedical Informatics Department, University of Colorado Anschutz Medical Center. October 2022.
\item Advanced Biomedical Computation Seminar, Computational Pathology Department, Brigham and Women's Hospital. March 2022.
\item Statistics and Data Science Department, Carnegie Mellon University. March 2022.
\item Biomedical Data Science Department, Stanford University. February 2022.
\item Biostatistics Department, University of Michigan. January 2022.
\item Population and Public Health Sciences Department, University of Southern California. January 2022.
\item Emerging Concepts in Microbial Informatics, Molecular Medicine and Biotechnology (panelist). Trinity University of Asia. November 2021.
\item Don't Normalize: the GLM-PCA approach to normalization. \href{https://normjam.github.io/}{{\it Normjam}}, New York Genome Center. November 2019.
\item Biostatistics and Data Science as a Career Path. Biology Department, Washington \& Lee University. October 2019.
\end{enumerate}

\textbf{Contributed Talks}
\begin{enumerate}[label= {[\arabic*]}]
\item Latent factorization methods for genomics. {\it New England Statistical Symposium} (virtual presenter). May 2022.
\item Nonnegative process factorization for multivariate spatial count data. {\it Joint Statistical Meetings} (virtual conference). August 2021.
\item Quantile normalization of single-cell RNA-seq read counts without unique molecular identifiers. {\it Joint Statistical Meetings} (virtual conference). August 2020.
\item Dimension reduction for massive single-cell datasets. {\it Bioconductor Annual Meeting} (virtual conference). July 2020. slides \doi{10.7490/f1000research.1118084.1}
\item Dimension Reduction for Single Cell RNA-Seq based on a Multinomial Model. {\it Joint Statistical Meetings} (Denver, CO). August 2019.
\item Varying-Censoring Aware Matrix Factorization for Single Cell RNA-Sequencing. \href{https://www.bioconductor.org/help/course-materials/2017/BioC2017/}{{\it Bioconductor Annual Meeting}} (Boston, MA). July 2017.
\end{enumerate}

\textbf{Seminars}
\begin{enumerate}[label= {[\arabic*]}]
\item Human tumor atlas network steering committee meeting. April 2022.
\item Nonnegative spatial factorization. {\it Hansen and Hicks labs, Department of Biostatistics, Johns Hopkins Bloomberg School of Public Health}. November 2021.
\item Primer: Generalized linear models and latent factor models. \href{https://www.broadinstitute.org/talks/primer-generalized-linear-models-and-latent-factor-models}{{\it Broad Institute Models, Inference, and Algorithms}}. October 2020.
%\item Feature Selection and Dimension Reduction for Single Cell RNA-Seq. {\it Engelhardt Lab, Computer Science Department, Princeton University}. February 2019.
\item Feature Selection and Dimension Reduction for Single Cell RNA-Seq. {\it Greene Lab, Department of Systems Pharmacology and Translational Therapeutics, Perelman School of Medicine, University of Pennsylvania}. February 2019.
\item Unsupervised Learning for Single Cell Gene Expression. {\it Harvard Biostatistics Cancer Working Group}. September 2018.
\item Single Cell Housekeeping Genes and Normalization. {\it Dana-Farber Cancer Institute cBio Seminar}. April 2018.
\item Informative Missing Data in Single Cell RNA-Seq. {\it Dana-Farber Cancer Institute Genomics Seminar}. October 2016.
\item Family Based Association Tests for Rare Variants. {\it Brigham \& Women's Hospital Channing Network Medicine Seminar}. September 2015.
\end{enumerate}
%\vskip .2in

\textbf{Interviews}
\begin{enumerate}[label={[\arabic*]}]
\item The Bioinformatics Chat. March 2020. \url{https://bioinformatics.chat/glm-pca}
\end{enumerate}

%\pagebreak

\textbf{Posters}
\begin{enumerate}[label= {[\arabic*]}]
\item Townes FW, Shukla C. Gene Expression Autoencoders. Harvard Biomedical Informatics 707 (Deep Learning in Healthcare). April 2018.
\item Townes FW, Marquez-Luna C. Mixture of Experts Analysis of Infectious Disease Outbreak Characteristics. Harvard Computer Science 281 (Advanced Machine Learning). December 2015.
\item Townes FW, Karaayvaz M, Gillespie S, Bernstein B, Ellisen L, Aryee M. Single Cell RNA-Seq Technical and Biological Confounders. Program in Quantitative Genomics Conference, Harvard Medical School. November 2015.
\end{enumerate}
%\vskip .2in

%\pagebreak
\vskip -.15in
\section{Peer Review}
\textbf{Journals}: {\it American Journal of Human Genetics, Bioinformatics, Biostatistics, Genome Biology,}
\newline{\it Journal of Machine Learning Research, NAR Genomics and Bioinformatics, Nature Communications}

%%%%%%%%%%%%%%%%%%%%%%%%%%%%%%%%%%%%%%
%%%%%%%%%%%%%% Awards %%%%%%%%%%%%%%%%%%
%%%%%%%%%%%%%%%%%%%%%%%%%%%%%%%%%%%%%%
\section{Professional Societies}
American Statistical Association, International Society for Computational Biology, Sigma Xi
\newpage

% \vskip -.15in
\section{Funding}
\vskip -0.25in
\begin{table}[!htb]
\begin{tabular}{rl}
2018 - 2019 & NIH T32 Training Grant: Cancer\\
2017 & Chan-Zuckerberg Foundation Travel Grant\\
    & Human Cell Atlas Jamboree. European Bioinformatics Institute, Hinxton, UK\\
2016 - 2018 & NIH T32 Training Grant: Big Data to Knowledge\\
2016 & NSF-CBMS Travel Award: Topology, Geometry, and Statistics\\
2014 - 2016 & NIH T32 Training Grant: Clinical Epidemiology of Lung Diseases
\end{tabular}
\end{table}

%%%%%%%%%%%%%%%%%%%%%%%%%%%%%%%%%%%%
%%%%%%%%% Teaching Experience %%%%%%%%%%%%%%%%
%%%%%%%%%%%%%%%%%%%%%%%%%%%%%%%%%%%%

% \newpage
\vskip -.2in
\section{Teaching Experience}
\vskip -.2in
\begin{table}[!htb]
\begin{tabular}{ll}
2021 Fall & \textbf{Guest lecture}, Advanced Computational Genomics (Princeton - \href{https://registrar.princeton.edu/course-offerings/course-details?term=1222&courseid=002125}{COS 597D}) \\
2020 Fall & \textbf{Guest lecture}, Computational Biology of Single Cells (Princeton - \href{https://registrar.princeton.edu/course-offerings/course-details?term=1212&courseid=009080}{COS 597F}) \\
2018 Fall & \textbf{Teaching Assistant}, Applied Regression (Harvard - BST 210)   \\
& Grade homeworks, hold office hours, and teach a weekly lab section.  \\
2018 Summer & \textbf{Co-Instructor}, StatStart (Harvard) \\
& Statistics program for under-represented high school students. \\
& Gave two interactive lectures on graphing data and regression. \\
2017 Fall & \textbf{Teaching Assistant}, Applied Bayesian Analysis (Harvard - BST 228)   \\
& Grade homeworks, provide solutions, and hold office hours. \\
2017 Summer & \textbf{Lead Instructor}, Introduction to Data Science (PARSE Ltd.) \\
& Nonprofit, week-long statistics program for high school students. \\
& Developed course material and gave interactive lectures using R. \\
2017 Spring & \textbf{Teaching Assistant}, Applied Longitudinal Analysis (Harvard - BST 226)   \\
& Grade homeworks and hold office hours. \\
2016 Fall & \textbf{Teaching Assistant}, Intro to Statistical Methods (Harvard - BST 201)   \\
& Grade homeworks, provide solutions, hold office hours, and teach a weekly lab section. \\
2016 Summer & \textbf{Project Mentor}, Pipelines into Biostatistics (Harvard) \\
& Program for under-represented undergraduates. \\
& Guided three students on a pharmacogenomics data analysis. \\
2016 Spring & \textbf{Teaching Assistant}, Rates and Proportions (Harvard - BST 210)   \\
& Grade homeworks, provide solutions, hold office hours, and teach a weekly lab section.
\end{tabular}
\end{table}
%\vskip .2in
%\pagebreak
%%%%%%%%%%%%%%%%%%%%%%%%%%%%%%%%%%%%
%%%%%%%%%% Technical Skills %%%%%%%%%%%%%%%%%%
%%%%%%%%%%%%%%%%%%%%%%%%%%%%%%%%%%%%
\vskip -.2in
\section{Software Packages}
\vskip -.2in
\begin{enumerate}[label= {[\arabic*]}]
\item \textbf{glmpca}: dimension reduction for non-normally distributed data. R package: \url{https://cran.r-project.org/package=glmpca}, python package: \url{https://pypi.org/project/glmpca/}
\item \textbf{scry}: Small-Count Analysis Methods for High-Dimensional Data. Bioconductor R package: \url{https://bioconductor.org/packages/release/bioc/html/scry.html}
\item \textbf{quminorm}: Quantile normalization of non-UMI single cell read counts. R package: \url{https://github.com/willtownes/quminorm}
\item \textbf{spatial-factorization}: Spatially-aware probabilistic factor models. Python package: \url{https://github.com/willtownes/spatial-factorization-py}
\end{enumerate}

\section{Technical Skills}
\textbf{Programming}: R, Python, SQL, Git, \LaTeX, Shell \\%Stata, Julia, Matlab,
\textbf{Operating Systems}: MacOS, Linux, Windows \\
\textbf{Machine Learning Frameworks}: Caret, Scikit-learn, Keras, Tensorflow, Stan%, JAGS

\end{document}

%%%%%%%%%%%%%%%%%%%%%%%%%% End CV Document %%%%%%%%%%%%%%%%%%%%%%%%%%%%%